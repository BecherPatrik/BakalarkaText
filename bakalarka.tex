%%%  Ukázkový text a dokumentace stylu pro text závěrečné (bakalářské a
%%%  diplomové) práce na KI PřF UP v Olomouci
%%%  Copyright (C) 2012 Martin Rotter, <rotter.martinos@gmail.com>
%%%  Copyright (C) 2014 Jan Outrata, <jan.outrata@upol.cz>


%%  Pro získání PDF souboru dokumentu je třeba tento zdrojový text v
%%  LaTeXu přeložit (dvakrát) programem pdfLaTeX.

%%  V případě použití programu BibLaTeX pro tvorbu seznamu literatury
%%  je poté ještě třeba spustit program Biber s parametrem jméno
%%  souboru zdrojového textu bez přípony a následně opět (dvakrát)
%%  přeložit zdrojový text programem pdfLaTeX.

%%  Postup získání Postscriptového souboru je popsán v dokumentaci.


%%  Třída dokumentu implementující styl pro závěrečnou práci. Vybrané
%%  nepovinné parametry (ostatní v dokumentaci):

%%  'master' pro sazbu diplomové práce, jinak se sází bakalářská práce

%%  'field=kód' pro Váš studijní obor, kódy pro diplomovou práci 'uvt'
%%  pro Učitelství výpočetní techniky pro střední školy a 'binf' pro
%%  Bioinformatiku, jinak je výchozí Informatika, a pro bakalářskou
%%  práci 'ainfk' pro Aplikovanou informatiku v kombinované formě,
%%  'inf' pro Informatiku, 'infv' pro Informatiku pro vzdělávání a
%%  'binf' pro Bioinfomatiku, jinak je výchozí Aplikovaná informatika
%%  v prezenční formě

%%  'printversion' pro sazbu verze pro tisk (nebarevné logo a odkazy,
%%  odkazy s uvedením adresy za odkazem, ne odkazy do rejstříku),
%%  jinak verze pro prohlížeč

%%  'biblatex' pro zapnutí podpory pro sazbu bibliografie pomocí
%%  BibLaTeXu, jinak je výchozí sazba v prostředí thebibliography

%%  'language=jazyk' pro jazyk práce, jazyky english pro anglický,
%%  slovak pro slovenský, jinak je výchozí czech pro český

%%  'font=sans' pro bezpatkový font (Iwona Light), jinak výchozí
%%  patkový (Latin Modern)

\documentclass[
%  master,
%  field=inf,
%  printversion,
  biblatex=false,
%  language=english,
  font=serif,
  glossaries=false,
  tables=false,
  theorems=false,
  index
]{kidiplom}

%% Informace pro úvodní strany. V jazyku práce (pokud není v komentáři
%% uvedeno česky) a anglicky. Uveďte všechny, u kterých není v
%% komentáři uvedeno, že jsou volitelné. Při neuvedení se použijí
%% výchozí texty. Text pro jiný než nastavený jazyk práce (nepovinným
%% parametrem language makra \documentclass, výchozí český) se zadává
%% použitím makra s uvedením jazyka jako nepovinného parametru.

%% Název práce, česky a anglicky. Měl by se vysázet na jeden řádek.
\title{Demonstrace práce s datovými strukturami}
\title[english]{Data Structure Demonstration}

%% Volitelný podnázev práce, česky a anglicky. Měl by se vysázet na
%% jeden řádek. Výchozí je prázdný.
%% \subtitle{Demonstrace práce s datovými strukturami}
%% \subtitle[english]{Data Structure Demonstration}

%% Jméno autora práce. Makro nemá nepovinný parametr pro uvedení
%% jazyka.
\author{Patrik Becher}

%% Jméno vedoucího práce (včetně titulů). Makro nemá nepovinný
%% parametr pro uvedení jazyka.
\supervisor{Mgr. Tomáš Kühr, Ph.D.}

%% Volitelný rok odevzdání práce. Výchozí je aktuální (kalendářní)
%% rok. Makro nemá nepovinný parametr pro uvedení jazyka.
%\yearofsubmit{\the\year}

%% Anotace práce, včetně anglické (obvykle překlad z jazyka
%% práce). Jeden odstavec!
\annotation{Cílem bakalářské práce bylo vytvořit nástroj pro podporu výuky algoritmizace, konkrétně práce se základními stromovými datovými strukturami (binární vyhledávací stromy, AVL stromy, červenočerné stromy). Výsledná aplikace podporuje vizualizaci vybraných datových struktur, včetně názorné demonstrace běžně prováděných operací s těmito datovými strukturami se souběžným zobrazením pseudokódu prováděné operace.}

\annotation[english]{Toto mně doplní Míša... :D}

%% Klíčová slova práce, včetně anglických. Oddělená (obvykle) středníkem.
\keywords{Binární vyhledávací stromy, Binární strom, AVL strom, Červenočerný strom, Stromové animace Java, JavaFX}
\keywords[english]{Binary search trees, Binary Tree, AVL tree, Redblack tree, Tree animations, Java, JavaFX}

%% Volitelná specifikace příloh textu práce, i anglicky. Výchozí je '1
%% CD/DVD'.
%\supplements{jedno kulaté placaté CD/DVD s malou kulatou dírou uprostřed}
%\supplements[english]{one round flat CD/DVD with a small round hole in the middle}

%% Poděkování. Stručné!
\thanks{Rád bych poděkoval panu Mgr. Tomáši Kührovi Ph.D. za vedení této bakalářské práce a panu RNDr. Arnoštu Večerkovi za odbornou pomoc a poskytnuté materiály k práci. Dále bych poděkoval mé rodině a přítelkyni za podporu při tvorbě.}

%% Další dodatečné styly (balíky) potřebné pro sazbu vlastního textu práce.
\usepackage{lipsum}

%% Balíčky pro implementaci seznamů vedle sebe.
\usepackage{amssymb}
\usepackage{enumitem}

%-------------------------------------------------------------
% TODO: Nebylo by špatné barvy a
% zvýraznění syntaxe jazyka GLSL přesunout do
% samostatného souboru.
\usepackage{color}

%-------------------------------------------------------------
% Nadefinuj barvy
\definecolor{backcolour}{rgb}{0.95,0.95,0.92}

%-------------------------------------------------------------
% Nadefinuj zvýraznění syntaxe pro jazyk Pseudo
% TODO: 1) Definice není úplně čistá.
%            2) Přesuň tuto definici do vlastního souboru.
\lstdefinelanguage{Pseudo}
{
  sensitive=false, % keywords are not case-sensitive
  keywords={define, end, return, while, do, if, then, else, elseif, for, to, do, end, relief, paraOcc, loadSample},
  morekeywords={},
  %alsoletter=\#\_, % now I can use # character in keywords
  keywordstyle=\bfseries\color{black}, % style of keywords
  captionpos=b, % Position of the Caption (t for top, b for bottom)
  extendedchars=true, % Allows 256 instead of 128 ASCII characters
  tabsize=2, % number of spaces indented when discovering a tab 
  columns=fixed, % make all characters equal width
  keepspaces=true, % does not ignore spaces to fit width, convert tabs to spaces
  showstringspaces=false, % lets spaces in strings appear as real spaces
  breaklines=true, % wrap lines if they don't fit
  %numbers=left, % show line numbers at the left
  %numberstyle=\ttfamily, % style of the line numbers
  backgroundcolor=\color{backcolour},
  basicstyle=\small\color{black},
  commentstyle=\color{green}, % style of comments
  stringstyle=\color{black}, % style of strings
  morecomment=[l]{//}, % l is for line comment
  morecomment=[s]{/*}{*/}, % s is for start and end delimiter
  morestring=[b]" % defines that strings are enclosed in double quotes
  moredelim=[is][\bfseries]{/@}{@/}
}
%-------------------------------------------------------------

% \noindent pro cely dokument.
% \setlength{\parindent}{0pt}

%-------------------------------------------------------------
\begin{document}
%% Sazba úvodních stran -- titulní, s bibliografickými údaji, s
%% anotací a klíčovými slovy, s poděkováním a prohlášením, s obsahem a
%% se seznamy obrázků, tabulek, vět a zdrojových kódů (pokud jejich
%% sazba není vypnutá).

%-------------------------------------------------------------
% Strany 1 - 6

\maketitle

%% Vlastní text závěrečné práce. Pro povinné závěry, před přílohami,
%% použijte prostředí kiconclusions. Povinná je i příloha s obsahem
%% přiloženého CD/DVD.

%% -------------------------------------------------------------------

%% Zatim neoddelat zjistit co to je.
% \newcommand{\BibLaTeX}{\textsc{Bib}\LaTeX}

%-------------------------------------------------------------
% Strana 7

\section{Úvod}
Tato aplikace vznikla za účelem výuky základních binárních vyhledávacích stromů. Obsahuje podporu pro binární, AVL a červenočerné stromy. Program pomocí animací zobrazuje operace: \textit{Vyhledávání}, \textit{Vkládání} a \textit{Odebírání} prvků ze stromů. Souběžně s animací zobrazuje stručný pseudokód aktuálně prováděné operace. Dále umožňuje \textit{Opakovat poslední operaci} a \textit{Generování náhodných stromů}.\\

Text samotné práce se dělí na dvě části: \textit{Teoretickou část}, ve které se zabývám teorií vybraných stromů a \textit{Programovou část}, která popisuje samotnou implementaci a funkcionalitu programu.\\

\newpage
\section{Stromy}
V kapitole jsou vysvětleny základní pojmy, které jsou nezbytné k pochopení vlastností \textit{stromů} obsažených v aplikaci. V podkapitolách jsou osvětleny principy pro tvorbu a následnou práci s konkrétními \textit{binárními vyhledávacími stromy}. Využitím těchto principů, byl naprogramován tento výukový nástroj.\\

\smallskip
\begin{definition}[Strom]
\textit{Strom} je neorientovaný \footnote{Mezi každými dvěma vrcholy existuje právě jedna cesta.} souvislý \footnote{Vynecháním libovolné hrany vznikne nesouvislý graf.} graf bez kružnic \footnote{Přidáním jakékoli hrany vznikne graf s kružnicí.} \cite{belohlavekALM}.
\end{definition}
\begin{figure}[h!]
\centering
	\includegraphics[scale=0.6]{obrazky/1Stromy.png}
	\caption{Příklady neorientovaných stromů}
	\label{BVS}
\end{figure}

\medskip
Strom je datová struktura, která představuje stromovou strukturu propojených \textit{uzlů} \footnote{Prvek obsahující hodnotu.}. Uzly jsou mezi sebou vzájemně spojeny pomocí \textit{hran} \footnote{Představuje cestu mezi spojenými uzly.}. 

\begin{definition}[Kořenový strom]
\textit{Kořenový strom} je strom, ve kterém je vybrán jeden vrchol (kořen). Může to být kterýkoliv vrchol. Bývá to ale vrchol, který je v nějakém smyslu na vrcholu hierarchie objektů, která je stromem reprezentována.
\cite{belohlavekALM}
\end{definition}
\smallskip

\newpage
\noindent {\large\textbf{Důležité pojmy:}}
\begin{itemize}
\item \textbf{Kořen} -- Jeden uzel, který se nachází na vrcholu stromu. Tento uzel nemá \textit{rodiče}. 
\item \textbf{Potomek, následník} -- Uzel, který je přímo připojen k jinému uzlu, cestou od kořene.
\item \textbf{Rodič, předchůdce} -- Uzel bez potomků.
\item \textbf{Sourozenci} -- Skupina uzlů, které mají stejného rodiče.
\item \textbf{Koncový uzel, list} -- Uzel, který nemá žádného potomka. 
\end{itemize} 

\subsection{Binární stromy}
\begin{definition}[Binární stromy]
\textit{Binární stromy} jsou typy stromů, ve kterých každý obsažený uzel má maximálně 2 potomky.
\end{definition}
\smallskip

\noindent\textbf{Každý uzel obsahuje tyto vlastnosti:}
\begin{itemize}
\item \textbf{Klíč} -- Hodnota uložená v uzlu.
\item \textbf{Ukazatel na levého potomka.}
\item \textbf{Ukazatel na pravého potomka.}
\item \textbf{Ukazatel na jednoho rodiče.}
\end{itemize}
 
%% Závěry práce. V jazyce práce a anglicky. Text pro jiný než
%% nastavený jazyk práce (nepovinným parametrem language makra
%% \documentclass, výchozí český) se zadává použitím makra s uvedením
%% jazyka jako nepovinného parametru.
\begin{kiconclusions}
Závěr práce v \uv{českém} jazyce.
\end{kiconclusions}

\begin{kiconclusions}[english]
Thesis conclusions in \uv{English}.
\end{kiconclusions}

%% Přílohy obsahu textu práce, za makrem \appendix.
\appendix

\section{První příloha}
Text první přílohy

\section{Druhá příloha}
Text druhé přílohy

%% Obsah přiloženého CD/DVD. Poslední příloha. Upravte podle vlastní
%% práce!
\section{Obsah přiloženého CD/DVD} \label{sec:ObsahCD}

Na samotném konci textu práce je uveden stručný popis obsahu
přiloženého CD/DVD, tj.~jeho závazné adresářové struktury, důležitých
souborů apod.

\begin{description}

\item[\texttt{bin/}] \hfill \\
  Instalátor \textsc{Instalator} programu, popř.~program
  \textsc{Program}, spustitelné přímo z~CD/DVD. / Kompletní adresářová
  struktura webové aplikace \textsc{Webovka} (v~ZIP archivu) pro
  zkopírování na webový server. Adresář obsahuje i~všechny runtime
  knihovny a~další soubory potřebné pro bezproblémový běh instalátoru
  a~programu z~CD/DVD / pro bezproblémový provoz webové aplikace na
  webovém serveru.

\item[\texttt{doc/}] \hfill \\
  Text práce ve formátu PDF, vytvořený s~použitím závazného stylu KI
  PřF UP v~Olomouci pro závěrečné práce, včetně všech příloh,
  a~všechny soubory potřebné pro bezproblémové vygenerování PDF
  dokumentu textu (v~ZIP archivu), tj.~zdrojový text textu, vložené
  obrázky, apod.

\item[\texttt{src/}] \hfill \\
  Kompletní zdrojové texty programu \textsc{Program} / webové aplikace
  \textsc{Webovka} se všemi potřebnými (příp.~převzatými) zdrojovými
  texty, knihovnami a~dalšími soubory potřebnými pro bezproblémové
  vytvoření spustitelných verzí programu / adresářové struktury pro
  zkopírování na webový server.

\item[\texttt{readme.txt}] \hfill \\
  Instrukce pro instalaci a~spuštění programu \textsc{Program}, včetně
  všech požadavků pro jeho bezproblémový provoz. / Instrukce pro
  nasazení webové aplikace \textsc{Webovka} na webový server, včetně
  všech požadavků pro její bezproblémový provoz, a~webová adresa, na
  které je aplikace nasazena pro účel testování při tvorbě posudků
  práce a~pro účel obhajoby práce.

\end{description}

Navíc CD/DVD obsahuje:

\begin{description}

\item[\texttt{data/}] \hfill \\
  Ukázková a~testovací data použitá v~práci a~pro potřeby testování
  práce při tvorbě posudků a~obhajoby práce.

\item[\texttt{install/}] \hfill \\
  Instalátory aplikací, runtime knihoven a~jiných souborů potřebných
  pro provoz programu \textsc{Program} / webové aplikace
  \textsc{Webovka}, které nejsou standardní součástí operačního
  systému určeného pro běh programu / provoz webové aplikace.

\item[\texttt{literature/}] \hfill \\
  Vybrané položky bibliografie, příp.~jiná užitečná literatura
  vztahující se k~práci.

\end{description}

U~veškerých cizích převzatých materiálů obsažených na CD/DVD jejich
zahrnutí dovolují podmínky pro jejich šíření nebo přiložený souhlas
držitele copyrightu. Pro všechny použité (a~citované) materiály,
u~kterých toto není splněno a~nejsou tak obsaženy na CD/DVD, je uveden
jejich zdroj (např.~webová adresa) v~bibliografii nebo textu práce
nebo v souboru \texttt{readme.txt}.

%% -------------------------------------------------------------------

%% Sazba volitelného seznamu zkratek, za přílohami.
%\printglossary

%% Sazba povinné bibliografie, za přílohami (případně i za seznamem
%% zkratek). Při použití BibLaTeXu použijte makro
%% \printbibliography. jinak prostředí thebibliography. Ne obojí!

%% Sazba i v textu necitovaných zdrojů, při použití
%% BibLaTeXu. Volitelné.
\nocite{*}
%% Vlastní sazba bibliografie při použití BibLaTeXu.
%% \printbibliography


\newpage


\begin{thebibliography}{99}	

\bibitem{belohlavekALM} \uppercase{BĚlohlávek}, Radim. Algoritmická matematika 2 - část 1 [online]. 2012-05-15. [cit. 2018-07-07]. Dostupné z: \url{http://belohlavek.inf.upol.cz/vyuka/algoritmicka-matematika-2-1.pdf}

\bibitem{belohlavekVilem} \uppercase{BĚlohlávek}, Radim; \uppercase{Vychodil}, Vilém; Diskrétní matematika pro informatiky II [online]. 2010-10-16. [cit. 2018-07-07]. Dostupné z: \url{http://belohlavek.inf.upol.cz/vyuka/dm2.pdf}

\bibitem{dvorsky} \uppercase{DvorskÝ}, Jiří. Algoritmy I.[online]. 2007-02-27. [cit. 2018-07-07]. Dostupné z: \url{http://www.cs.vsb.cz/dvorsky/Download/SkriptaAlgoritmy/Algoritmy.pdf}

\end{thebibliography}
%% Bibliografie, včetně sazby, při nepoužití BibLaTeXu.
% \begin{thebibliography}{9}
%\bibitem{kniha2} \uppercase{Hawke}, Paul. NanoHttpd: Light-weight HTTP server designed for embedding in other applications. GitHub [online]. 2014-05-12. [cit. 2014-12-06]. Dostupné z: \url{https://github.com/NanoHttpd/nanohttpd}
%
%\bibitem{jeske13} \uppercase{Jeske}, David; \uppercase{Novák}, Josef. Simple HTTP Server in \csharp: Threaded synchronous HTTP Server abstract class, to respond to HTTP requests. CodeProject: For those who code [online]. 2014-05-24. [cit. 2014-12-06]. Dostupné z: \url{http://www.codeproject.com/Articles/137979/Simple-HTTP-Server-in-C}
%
%\bibitem{uzis2012} \uppercase{ÚSTAV ZDRAVOTNICKÝCH INFORMACÍ A STATISTIKY ČR}. Lékaři, zubní lékaři a farmaceuti 2012 [online]. Praha 2, Palackého náměstí 4: Ústav zdravotnických informací a statistiky ČR, 2012 [cit. 2014-12-06]. ISBN 978-80-7472-089-5. Dostupné z: \url{http://www.uzis.cz/publikace/lekari-zubni-lekari-farmaceuti-2012}



%% Sazba volitelného rejstříku, za bibliografií.
\printindex

\end{document}
